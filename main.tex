\documentclass[a4paper,12pt]{article}
\usepackage{amsmath}
\usepackage{amssymb}
\usepackage{xcolor}
\usepackage{calc}
\usepackage[brazilian]{babel}
\usepackage[left=3cm,top=3cm,bottom=2cm,right=2cm]{geometry}

\title{\textbf{Cheat Sheet: FMC - Unidade 2}}
\author{Mateus Dias \\ Tecnologia da Informação - IMD/UFRN}
\date{\today}

\begin{document}

\maketitle

\tableofcontents
\newpage

\section{DEFINIÇÃO 0 - Divisores e Múltiplos}
Essa definição não faz, realmente, parte do conteúdo, mas é fundamental para o entendimento de todas as próximas definições.
\subsection{Divisores}
Um número inteiro $d$ é divisor de um número inteiro $a$ se, e somente se, ao dividir $a$ por $d$, o resto for \textbf{zero}, ou seja, a divisão é exata.\\
Por exemplo, para o número $24$ temos $8$ divisores. São eles:\\
\[
    D = \{1,2,3,4,6,8,12,24\}
\]
\subsection{Múltiplos}
Um número inteiro $b$ é múltiplo de um número inteiro $a$ se, e somente se, existe um número inteiro $k$ tal que:
\[
    b = ak
\]
Por exemplo, se $a = 3$, os múltiplos de 3 são:
\begin{itemize}
    \item Se $k = 2$, $b = 3 \cdot 2 = 6$
    \item Se $k = 5$, $b = 3 \cdot 5 = 15$
\end{itemize}
O conjunto dos múltiplos de $3$ é:
\[
    M(3) = \{\dots, -9, -6, -3, 0, 3, 6, 9, 12, \dots\}
\]
\subsection{Relação entre Divisores e Múltiplos}
Se $b$ é um \textbf{múltiplo} de $a$, isso significa que $a$ é um \textbf{divisor} de $b$. Esta relação será explorada melhor na definição de \textbf{divisibilidade}.
\vspace{6pt}\\

\section*{DEFINIÇÃO 1 - Divisibilidade}
Sejam $a, b \in \mathbb{Z}$. Dizemos que $a$ divide $b$ se, e somente se $\exists k \in \mathbb{Z}$ tal que: $ak = b$.
\[
    a \mid b \iff (\exists k \in \mathbb{Z}) (ak = b).
\]
\vspace{-5pt}\\

\section{DEFINIÇÃO 2 - Módulo}
Sejam $a, b, m \in \mathbb{Z}$. $a$ é congruente a $b$ módulo $m$ se, e somente se $m \mid a-b$.
\[
    a \equiv b \pmod m \iff m \mid a-b
\]
Também é possível usar o módulo para representar o resto de uma divisão. Pela definição de divisão euclidiana, sabe-se que um número arbitrário $D \in \mathbb{Z}$ pode ser representado como
\[
    D = dq+r\text{.}
\]
Com $\mathbf{D}$ sendo o \textbf{dividendo}, $\mathbf{d}$ o \textbf{divisor}, $\mathbf{q}$ o \textbf{quociente} e $\mathbf{r}$ o \textbf{resto} ($0 \leq r < |d|$).\\
Nesse sentido, podemos afirmar que:
\[
    D \bmod d = r\text{.}
\]
\vspace{6pt}\\

\section*{TEOREMA 1}
Sejam $a,b,m \in \mathbb{Z}$, com $m > 0$. $a$ é congruente a $b$ módulo $m$ se, e somente se $a \bmod m = b \bmod m$.
\[
     a \equiv b \pmod m \iff a \bmod m = b \bmod m \text{.}
\]
\subsection*{Prova ($\implies$)}
Suponha $a, b, m \in \mathbb{Z}$ tal que $a \equiv b \pmod m$.\\
Pelas definições 2 e 1, respectivamente, temos que:
\begin{flalign*}
      & m \mid a -b                          \\
      & mk = a-b                             \\
      & a = b + mk \quad \text{\textbf{(I)}}
\end{flalign*}
Pela definição do resto, ao dividir $b$ por $m$, temos:
\[
     b = q_b \cdot m + r_b \quad \text{\textbf{(II)}}
\]
Onde $q_b \in \mathbb{Z}$ e $r_b = b \bmod m$, com $0 \leq r_b < m$.\\
Substituindo \textbf{(II)} em \textbf{(I)}:
\begin{flalign*}
      & a = (q_b \cdot m + r_b) + mk \\
      & a = mq_b + mk + r_b          \\
      & a = m\cdot(q_b + k) + r_b
\end{flalign*}
Tome $q_a = q_b + k$. Como $q_a \in \mathbb{Z}$ e $0 \leq r_b < m$, podemos dizer que $r_b$ é o resto da divisão de $a$ por $m$, isto é, $a \bmod m = r_b$.\\
Como $r_b = b \bmod m$, temos que:
\[
     a \bmod m = b \bmod m \text{.}
\]
\subsection*{Prova ($\impliedby$)}
Suponha $a, b, m \in \mathbb{Z}$ tal que $a \bmod m = b \bmod m$.
Seja $r$ o valor comum do resto, de forma que:

\[
     r = a\bmod m = b \bmod m
\]

Pela definição do resto, podemos escrever $a$ e $b$ como:
\begin{flalign*}
      & a = q_a \cdot m + r \\
      & b = q_b \cdot m + r
\end{flalign*}

Onde $q_a, q_b \in \mathbb{Z}$ e $0 \leq r < m$.
Dessa forma, a diferença $a - b$ fica desta forma:
\begin{flalign*}
      & a - b = (q_a \cdot m + r) - (q_b \cdot m + r) \\
      & a - b = q_a \cdot m + r - q_b \cdot m - r     \\
      & a - b = m \cdot (q_a-q_b)
\end{flalign*}

Seja $k = q_a-q_b$. Como $q_a \in \mathbb{Z}$ e $q_b \in \mathbb{Z}$ temos que $k \in \mathbb{Z}$. Portanto:

\[
     a-b = mk
\]

Pela definição 1 e 2, respectivamente, temos que:
\begin{flalign*}
      & m \mid a-b         \\
      & a \equiv b \pmod m
\end{flalign*}

\begin{flalign*}
      &  & \blacksquare
\end{flalign*}

\section{DEFINIÇÃO 3 - Máximo Divisor Comum}
Sejam $a, b \in \mathbb{Z}$ com $a, b \neq 0$. O MDC de $a$ e $b$, denotado por $mdc(a, b)$ é o único inteiro positivo $d$ que satisfaz as seguintes condições:
\begin{enumerate}
    \item $d \mid a$
    \item $d \mid b$
    \item $\forall c \in \mathbb{Z} [((c \mid a) \land (c \mid b)) \implies c \mid d]$
\end{enumerate}

Em outros termos, $d$ é o maior número inteiro positivo que divide $a$ e $b$ ao mesmo tempo.

\subsection*{Exemplos}

\textbf{1.} Calcular o $mdc(12, 18)$.
\begin{align*}
     & \text{Divisores de 12: } \{1,2,3,4,6,12\} \\
     & \text{Divisores de 18: } \{1,2,3,6,9,18\} \\
     & \text{Divisores comuns: } \{1,2,3,6\}     \\
     & \textbf{Máximo Divisor Comum (MDC): } 6
\end{align*}
\vspace{-10pt}

\subsection{Algoritmo de Euclides}
O algoritmo de Euclides é um método simples para encontrar o MDC entre dois números inteiros diferentes de zero. Ele é um derivado da divisão euclidiana:
\[
    D = dq + r\text{.}
\]
Com $\mathbf{D}$ sendo o \textbf{dividendo}, $\mathbf{d}$ o \textbf{divisor}, $\mathbf{q}$ o \textbf{quociente} e $\mathbf{r}$ o \textbf{resto} ($0 \leq r < |d|$).\\\\
Se queremos calcular $mdc(a,b)$, podemos assumir $D_1 = \max(a,b)$ como o dividendo inicial e $d_1 = \min(a,b)$ como o divisor inicial.\\\\
O algoritmo procede em etapas sucessivas, onde o resto de cada divisão se torna o novo divisor e o divisor anterior se torna o novo dividendo, até que $r_i = 0$ (onde $i$ é o número de iterações). O último resto \textbf{não nulo} é o $mdc(a,b)$.

\subsection*{Exemplos}
\textbf{2.} Calcular o $mdc(270, 192)$.
\begin{align}
     & 270 = 192 \cdot 1 + 78 \\
     & 192 = 78 \cdot 2 + 36  \\
     & 78 = 36 \cdot 2 + 6    \\
     & 36 = 6 \cdot 6 + 0
\end{align}
Portanto, o $\mathbf{mdc(270,192)}$ é igual ao último resto não nulo, ou seja, $\mathbf{6}$.

\section*{DEFINIÇÃO 4 - Mínimo Múltiplo Comum}
Sejam $a, b \in \mathbb{Z}$. O $mmc(a,b)$ é o menor número inteiro positivo que é múltiplo de $a$ e $b$ simultaneamente.

\subsection*{Exemplos}
\textbf{1.} Calcular o $mmc(4,6)$.
\begin{align*}
    \textbf{Múltiplos de 4: } \{4,8,\mathbf{12},16,20,24,...\}  \\
    \textbf{Múltiplos de 6: } \{6,\mathbf{12},18,24,30,36,...\} \\
\end{align*}
O menor dos múltiplos comuns é $12$, portanto $mmc(4,6) = 12$.\\
\vspace{-10pt}\\

\subsection*{Métodos para calcular o MMC}
É possível conectar os conceitos de MMC e MDC com uma fórmula relacionada à Matemática Discreta:

\[
    mmc(a,b) = \frac{|a \cdot b|}{mdc(a,b)}
\]
\\
Este é o método que apresenta maior eficiência computacional para calcular o MMC entre dois números, mas também existe o método da fatoração prima (mais útil para calcular o MMC entre três ou mais números):

\begin{enumerate}
    \item \textbf{Fatore} todos os números em seus fatores primos;
    \item O \textbf{MMC} é o produto de todos os fatores primos distintos, cada um elevado à maior potência em que ele aparece em qualquer uma das fatorações.
\end{enumerate}

\subsection*{Exemplos}
\textbf{2.} Calcular o $mmc(12,18)$ usando o primeiro método.

\begin{enumerate}
    \item Calcular o \textbf{mdc(12,18)}:\\
          Segundo o método apresentado na \textbf{definição 3}:
          \begin{align}
              18 & = 12 \cdot 1 + 6 \tag{1} \\
              12 & = 6 \cdot 2 + 0 \tag{2}
          \end{align}
          Portanto, $mdc(12,18) = \mathbf{6}$.
    \item Substituir na fórmula:\\
          \begin{align*}
               & mmc(12,18) = \frac{|12 \cdot 18|}{6} \\
              \\
               & mmc(12,18) = \frac{216}{6}           \\
              \\
               & mmc(12,18) = 36
          \end{align*}
\end{enumerate}
\textbf{3.} Calcular o $mmc(12,18)$ usando o segundo método.

\begin{enumerate}
    \item Fatore 12 e 18 em seus respectivos fatores primos:
          \begin{align*}
               & 12 = 2^2 \cdot 3^1 \\
               & 18 = 2^1 \cdot 3^2
          \end{align*}
    \item Fatores e maiores potências:
          \begin{itemize}
              \item Fator 2: $2^2$
              \item Fator 3: $3^2$
          \end{itemize}
    \item Cálculo:
          \begin{align*}
               & mmc(12,18) = 2^2 \cdot 3^2 \\
               & mmc(12,18) = 4 \cdot 9     \\
               & mmc(12,18) = 36
          \end{align*}
\end{enumerate}

O MMC entre $a$ e $b$ também pode ser interpretado como ``o primeiro número em que $a$ irá \textit{se encontrar} com $b$ quando ambos forem multiplicados por números naturais''.

\section{TEOREMA 2 - Teorema de Bézout}
Sejam $a, b \in \mathbb{Z}$ com $a,b > 0$. O $mdc(a,b)$ pode ser escrito como uma combinação linear de $a$ e $b$:
\[
    mdc(a,b) = sa + tb
\]
Com $s,t \in \mathbb{Z}$.\\
O método para descobrir os valores de $s$ e $t$ é substituir consecutivamente os valores no algoritmo de Euclides.

\subsection*{Exemplos}
\textbf{1.} Expressar o mdc(270,192) como uma combinação linear de 270 e 192.
\begin{align*}
     & 270 = 192 \cdot 1 + 78                                                \\
     & 192 = 78 \cdot 2 + 36                                                 \\
     & 78 = 36 \cdot 2 + 6                                                   \\
     & 36 = 6 \cdot 6 + 0                                                    \\
    \\
     & 6 = 78 - 2 \cdot 36                                                   \\
     & 6 = 78 - 2 \cdot (192 - 2 \cdot 78)                                   \\
     & 6 = (270 - 1 \cdot 192) - 2 \cdot (192 - 2 \cdot (270 - 1 \cdot 192)) \\
     & 6 = 270 -  1 \cdot 192 - 2 \cdot 192 + 4 \cdot 270 - 4 \cdot 192      \\
     & 6 = 5 \cdot 270 - 7 \cdot 192
\end{align*}
\par Portanto, $s = 5, t = -7$.
\vspace{6pt}\\

\section{DEFINIÇÃO 5 - Inverso Multiplicativo Modular}
Este é um conceito essencial que se relaciona com o conceito de congruência linear (\textbf{definição 6}).\\\\
Sejam $a,m \in \mathbb{Z}$. O inverso multiplicativo modular de $a \bmod m$ é o inteiro $x$ tal que:
\[
    ax \equiv 1 \pmod m
\]

\subsection{Condição de existência}
O inverso multiplicativo modular de $a \bmod m$ existe se, e somente se $mdc(a,m) = 1$, isto é, se $a$ e $m$ forem \textbf{coprimos} ou \textbf{primos entre si}.\\

\subsection{Métodos para encontrar}
É possível encontrar o inverso multiplicativo de $a \bmod m$ facilmente usando o \textbf{teorema 2 - teorema de Bézout}.
\\\\Ao escrever o $mdc(a,m)$ como uma combinação linear de $a$ e $m$, o coeficiente de $a$ é o seu inverso multiplicativo.

\subsection*{Exemplos}
\textbf{1.} Encontrar o inverso multiplicativo de $3 \bmod 7$.

\begin{align*}
     & 3x \equiv 1 \pmod 7                                             \\
    \\
     & \text{Primeiro, precisamos calcular $mdc(3,7)$.}                \\
    \\
     & 7 = 3 \cdot 2 + 1                                               \\
     & 3 = 1 \cdot 3 + 0                                               \\
    \\
     & \text{Como $mdc(3,7) = 1$, o inverso multiplicativo existe.}    \\
     & \text{Agora, escrevemos 1 como uma combinação linear de 3 e 7.} \\
    \\
     & 1 = 1 \cdot 7 - 2 \cdot 3                                       \\
    \\
     & \text{O coeficiente de 3 é -2, então $x = -2$.}
\end{align*}

Como o inverso multiplicativo encontrado é um número negativo, podemos fazer a operação $x \bmod m$ para encontrar um inverso multiplicativo positivo (o que é uma boa prática).

\[
    -2 \bmod 7 = 5
\]

Logo, o inverso multiplicativo que procuramos é $\mathbf{5}$.\\

\textbf{Obs:} Para encontrar um inverso multiplicativo positivo também é possível somar $m$ a $x$ até que $x$ seja maior ou igual a 1.

\section*{DEFINIÇÃO 6 - Congruência Linear}
Uma congruência linear é uma equação na forma $ax \equiv b \pmod m$. Uma congruência linear tem solução se, e somente se $mdc(a,m) \mid b$.\\
\\
Para resolver a congruência linear, é necessário seguir os seguintes passos:

\begin{enumerate}

    \item Encontrar o inverso multiplicativo de $a$ -- denotado por $\overline{a}$ ou $a^{-1}$ -- utilizando o método descrito na \textbf{definição 5}.
          \[
              a \overline{a} \equiv 1 \pmod m
          \]

    \item Multiplicar os dois lados da congruência por $\overline{a}$.
          \[
              \overline{a}ax \equiv \overline{a}b \pmod m
          \]
    \item Simplificando, o resultado fica:
          \[
              x \equiv \overline{a}b \pmod m
          \]
    \item Se $\overline{a}b < 1$ ou $\overline{a}b \geq m$, é necessário executar a operação $(\overline{a}b \bmod m)$ para encontrar a menor congruência natural.

\end{enumerate}

\subsection*{Exemplos}
\textbf{1.} Calcular $17x \equiv 82 \pmod{11}$

\begin{align*}
     & 17\overline{a} \equiv 1 \pmod{11}                                \\
    \\
     & 17 = 11 \cdot 1 + 6                                              \\
     & 11 = 6 \cdot 1 + 5                                               \\
     & 6 = 5 \cdot 1 + 1                                                \\
    \\
     & 1 = 6 - 1 \cdot 5                                                \\
     & 1 = 6 - 1 \cdot (11 - 1 \cdot 6)                                 \\
     & 1 = (17 - 1 \cdot 11) - 1 \cdot (11 - 1 \cdot (17 - 1 \cdot 11)) \\
     & 1 = 17 - 1 \cdot 11 - 1 \cdot 11 + 1 \cdot 17 -1 \cdot 11        \\
     & 1 = 2 \cdot 17 - 3 \cdot 11                                      \\
    \\
     & \overline{a} = 2                                                 \\
    \\
     & 2 \cdot 17x \equiv 2 \cdot 82 \pmod {11}                         \\
     & 34x \equiv 164 \pmod {11}                                        \\
     & \boxed{x \equiv 10 \pmod {11}}                                   \\
\end{align*}

Também é possível escrever a solução na forma de um conjunto solução, usando as definições 2 e 1 \textbf{(módulo e divisibilidade)}, respectivamente:
\begin{align*}
    11  & \mid x - 10                                                \\
    11k & = x - 10                                                   \\
    x   & = 11k + 10                                                 \\
    \\
    S   & = \{x \in \mathbb{Z} \mid x = 10 + 11k, k \in \mathbb{Z}\}
\end{align*}

\section{TEOREMA 3 - Teorema Chinês do Resto}
O teorema chinês do resto é um teorema que pode ser usado para resolver sistemas de congruências lineares do tipo:
\[
    \begin{cases}
        x \equiv a_1 \pmod{m_1}         \\
        x \equiv a_2 \pmod{m_2}         \\
        \dots                           \\
        x \equiv a_{n-1} \pmod{m_{n-1}} \\
        x \equiv a_n \pmod{m_n}
    \end{cases}
\]
\textbf{Obs:} A solução só existe se $mdc(m_i, m_j) = 1$ para todo $i \neq j$.

\subsection{Algoritmo de solução}
\begin{enumerate}
    \item Calcular o módulo total ($M$):
          \[
              M = m_1 \cdot m_2 \cdot \dots \cdot m_n
          \]
    \item Calcular os $M_i$:
          \[
              \text{$M_i$ é o produto de todos os módulos do sistema, excluindo o módulo $m_i$.}
          \]
    \item Encontrar o inverso $y_i$:
          \[
              M_i \cdot y_i \equiv 1 \pmod{m_i}
          \]
    \item Calcular a solução ($x$):
          \[
              x \equiv a_1 M_1 y_1 + a_2 M_2 y_2 + \dots + a_n M_n y_n \pmod M
          \]

          \textbf{Obs:} Na maioria das vezes, a solução final será o resto da divisão dessa soma por $M$.
\end{enumerate}

\subsection*{Exemplos}
\textbf{1.} Calcular a solução de:
\[
    \begin{cases}
        x \equiv 2 \pmod 3 \\
        x \equiv 3 \pmod 5 \\
        x \equiv 2 \pmod 7 \\
    \end{cases}
\]

\begin{enumerate}
    \item Módulo total ($\mathbf{M}$):
          \begin{align*}
              M & = 3 \cdot 5 \cdot 7 \\
              M & = 105
          \end{align*}
    \item Cálculo dos $\mathbf{M_i}$:
          \begin{align*}
              M_1 & = 5 \cdot 7 = 35 \\
              M_2 & = 3 \cdot 7 = 21 \\
              M_3 & = 3 \cdot 5 = 15 \\
          \end{align*}
    \item Inverso $\mathbf{y_i}$:
          \begin{align*}
              35 \cdot y_1 & \equiv 1 \pmod 3 \\
              21 \cdot y_2 & \equiv 1 \pmod 5 \\
              15 \cdot y_3 & \equiv 1 \pmod 7 \\
              \\
              y_1          & = 2              \\
              y_2          & = 1              \\
              y_3          & = 1
          \end{align*}
    \item Solução $\mathbf{x}$:
          \begin{align*}
              x & \equiv (2 \cdot 35 \cdot 2) + (3 \cdot 21 \cdot 1) + (2 \cdot 15 \cdot 1) \pmod{105} \\
              x & \equiv 140 + 63 + 30 \pmod {105}                                                     \\
              x & \equiv 233 \pmod {105}                                                               \\
          \end{align*}
          \[
              \boxed{
                  x \equiv 23 \pmod {105}
              }
          \]
\end{enumerate}

\section{TEOREMA 4 - Pequeno Teorema de Fermat}
Este é um teorema desenvolvido pelo matemático francês Pierre de Fermat, e possibilita a simplificação de cálculos com potências grandes na aritmética modular.\\
O teorema pode ser enunciado de duas formas:

\subsubsection*{Primeira forma}
Seja $p$ um número primo e $a \in \mathbb{Z}$. Então:
\[
    a^p \equiv a \pmod{p}
\]
Ou seja, se elevarmos $a$ à potência do primo $p$ e depois dividirmos o resultado por $p$, o resto é igual ao resto da divisão de $a$ por $p$.

\subsubsection*{Segunda forma}
Esta é a forma mais usada.\\
Seja $p$ um número primo e $a \in \mathbb{Z}$ tal que $p \nmid a$ (ou seja, $mdc(a,p) = 1$). Então:
\[
    a^{p-1} \equiv 1 \pmod {p}
\]
Ou seja, se elevarmos $a$ à potência de $p-1$ e depois dividirmos o resultado por $p$, o resultado vai sempre ser igual a $1$.

\subsection*{Exemplos}
\textbf{1.} Calcular $2^{23} \bmod 5$ (ou seja, o resto da divisão de $2^{23}$ por $5$).

\[
    \text{Como 5 é um número primo e $5 \nmid 2$, é possível utilizar o pequeno teorema de Fermat.}
\]

\setcounter{equation}{0}
\begin{align}
    2^{5-1}                 & \equiv 1 \pmod{5}                 \\
    2^{4}                   & \equiv 1 \pmod {5}                \\
    (2^{4})^{5}             & \equiv 1^{5} \pmod {5}            \\
    (2^{4})^{5} \cdot 2^{3} & \equiv 1^{5} \cdot 2^{3} \pmod{5} \\
    2^{23}                  & \equiv 8 \pmod{5}                 \\
    2^{23}                  & \equiv 3 \pmod {5}
\end{align}
\[
    \text{Portanto, o resto da divisão de $2^{23}$ por 5 é igual a 3.}
\]

\section{APLICAÇÃO 1 - Números Inteiros Grandes}
Na área da computação, muitas vezes é necessário computar números inteiros grandes que, a princípio, não podem ser computados por um processador comum de computador.\\

\subsection{Codificação}
Sejam $m_1, m_2, \dots, m_n$ inteiros maiores que 1 e primos entre si, com $m$ sendo o produto entre eles e  $a \in \mathbb{Z}$ tal que $0 \leq a < m$. É possível representar todos os números $a$ como a $n$-upla:
\[
    a = (a \bmod m_1, a \bmod m_2, \dots, a \bmod m_n).
\]
\\
Por exemplo, se definirmos $m_1 = 3$, $m_2 = 5$, teremos as seguintes representações:

\begin{align*}
     & 0 = (0, 0) & \qquad & 5 = (2, 0) & \qquad & 10 = (1, 0) \\
     & 1 = (1, 1) & \qquad & 6 = (0, 1) & \qquad & 11 = (2, 1) \\
     & 2 = (2, 2) & \qquad & 7 = (1, 2) & \qquad & 12 = (0, 2) \\
     & 3 = (0, 3) & \qquad & 8 = (2, 3) & \qquad & 13 = (1, 3) \\
     & 4 = (1, 4) & \qquad & 9 = (0, 4) & \qquad & 14 = (2, 4) \\
\end{align*}

\subsection{Decodificação}
Dada uma $n$-upla e seus $m_i$, é possível chegar facilmente ao valor representado usando o \textbf{teorema chinês do resto} (teorema 3).\\

Como exemplo, podemos tentar descobrir o valor representado por uma dupla aleatória do exemplo de \textbf{codificação}. Vamos usar a dupla $\mathbf{(2,1)}$ e $m_1 = 3, m_2 = 5$:

\begin{align*}
     & x = (2,1) = (x \bmod 3, x \bmod 5)                            \\
    \\
     & \begin{cases}
           x \equiv 2 \pmod 3 \\
           x \equiv 1 \pmod 5 \\
       \end{cases}                                            \\
    \\
     & M = 3 \cdot 5 = 15                                            \\
    \\
     & M_1 = 5                                                       \\
     & M_2 = 3                                                       \\
    \\
     & y_1 = 2                                                       \\
     & y_2 = 2                                                       \\
    \\
     & x \equiv (2 \cdot 5 \cdot 2) + (1 \cdot 3 \cdot 2) \pmod {15} \\
     & x \equiv 20 + 6 \pmod {15}                                    \\
     & x \equiv 26 \pmod {15}
\end{align*}
\[
    \boxed{x \equiv 11 \pmod {15}}
\]\\
\textbf{Obs:} Nesse caso, podemos considerar como verdadeira apenas a primeira equivalência, portanto $x = 11$.

\subsection{Operações aritméticas}
Para realizar operações aritméticas com as $n$-uplas, basta realizar tal operação entre o $i$-ésimo termo da primeira $n$-upla com o seu respectivo na segunda $n$-upla, e com o resultado realizar a operação módulo com o $m_i$ correspondente.

\subsubsection*{Restrições}
Para realizar a operação, o valor resultante deve poder ser escrito também como uma $n$-upla. Portanto, o resultado deve ser um dos possíveis valores de $a$ ($0 \leq a < m$).

\subsubsection*{Exemplo}
Sejam $a = 2, b = 3, m_1 = 3, m_2 = 5$.
\begin{align*}
    a + b & = 2 + 3 = (2,2) + (0,3)              \\
    a + b & = ((2 + 0) \bmod 3, (2 + 3) \bmod 5) \\
    a + b & = (2 \bmod 3, 5 \bmod 5)             \\
    a + b & = (2, 0)                             \\
    a + b & = 5
\end{align*}

\begin{align*}
    a \cdot b & = 2 \cdot 3 = (2,2) \cdot (0,3)              \\
    a \cdot b & = ((2 \cdot 0) \bmod 3, (2 \cdot 3) \bmod 5) \\
    a \cdot b & = (0 \bmod 3, 6 \bmod 5)                     \\
    a \cdot b & = (0, 1)                                     \\
    a \cdot b & = 6
\end{align*}

\vspace{20pt}

\noindent
\colorbox{yellow!30}{
    \begin{minipage}{\linewidth-2\fboxsep-2\fboxrule}
        \textbf{NOTA:}
        Aqui acabam os conteúdos da primeira prova da unidade 2 (que foi dividida em duas partes, sendo a primeira no dia \textbf{15 de outubro}).
    \end{minipage}
}

\vspace{20pt}

\section*{DEFINIÇÃO 7 - Função de Euler}
A função de Euler, função totiente ou função phi é fundamental para os próximos tópicos, especialmente o de criptografia de chave pública ou assimétrica.
\begin{gather*}
    \text{Notação:}\\
    \phi(n)
\end{gather*}
Ela calcula a quantidade de números naturais não nulos $m$ menores ou iguais a $n$ tais que $m$ e $n$ são relativamente primos, ou seja, $mdc(m,n) = 1$

\subsection*{Exemplos}
\textbf{1.} Calcular $\phi(6)$:

\begin{itemize}
    \item Os números $m \leq 6$ são $\{1,2,3,4,5,6\}$.
          \begin{itemize}
              \item $mdc(1,6) = 1$ \textbf{(coprimo)}
              \item $mdc(2,6) = 2$ (não coprimo)
              \item $mdc(3,6) = 3$ (não coprimo)
              \item $mdc(4,6) = 2$ (não coprimo)
              \item $mdc(5,6) = 1$ \textbf{(coprimo)}
              \item $mdc(6,6) = 6$ (não coprimo)
          \end{itemize}
    \item Os coprimos são $\mathbf{1}$ e $\mathbf{5}$, então $\phi(6) = 2$.
\end{itemize}
\vspace{12pt}
\textbf{2.} Calcular $\phi(7)$:

\begin{itemize}
    \item Como 7 é um número primo, todos os números naturais não nulos até $7-1$ são coprimos com ele, portanto $\phi(7)=6$.
\end{itemize}

\subsection*{Principais fórmulas}
Existem algumas fórmulas e propriedades que ajudam a calcular $\phi(n)$. Sejam $p,q,m,n \in \mathbb{Z}$ com $p \neq q$, $p$ e $q$ números primos e $m$ e $n$ coprimos.

\setcounter{equation}{0}
\begin{align}
     & \phi(p) = p-1                           \\
     & \phi(p \cdot q) = (p-1)\cdot(q-1)       \\
     & \phi(p^{2}) = p \cdot (p-1)             \\
     & \phi(m \cdot n) = \phi(m) \cdot \phi(n)
\end{align}

\subsection*{Exemplos}
\textbf{3.} Calcular $\phi(19)$:

\begin{itemize}
    \item Pela fórmula \textbf{(1)}, $\phi(7) = 7 - 1 = 6$.
\end{itemize}
\vspace{12pt}
\textbf{4.} Calcular $\phi(35)$:

\begin{itemize}
    \item Decompondo 35 em fatores primos, obtemos $35 = 7 \cdot 5$.
    \item Pela fórmula \textbf{(2)}, $\phi(35) = \phi(7 \cdot 5) = (7 - 1) \cdot (5 - 1) = 6 \cdot 4 = 24$.
\end{itemize}
\vspace{12pt}
\textbf{5.} Calcular $\phi(49)$:

\begin{itemize}
    \item Sabe-se que $49 = 7^{2}$.
    \item Pela fórmula \textbf{(3)}, $\phi(49) = \phi(7^{2}) = 7 \cdot (7 - 1) = 7 \cdot 6 = 42$.
\end{itemize}

\vspace{12pt}

\section*{TEOREMA 5 - Teorema de Euler}
O teorema de Euler, também conhecido como teorema de Fermat-Euler, é uma generalização do Pequeno Teorema de Fermat \textbf{(teorema 4)}. Seu enunciado é o seguinte:

\begin{gather*}
    \text{Sejam $a$ e $n$ dois inteiros positivos, onde $n>1$. Se $mdc(a,n) = 1$, então:}\\
    a^{\phi(n)} \equiv 1 \pmod {n}\text{.}
\end{gather*}

\vspace{12pt}

\subsection*{Exemplos}
\textbf{1.} Calcular $7^{100} \bmod 10$.

\begin{itemize}
    \item Como $mdc(7,10) = 1$, a condição é satisfeita, portanto, $7^{\phi(10)} \equiv 1 \pmod {10}$.
    \item Agora precisamos calcular $\phi(10)$.
          \begin{align*}
              \phi(10) & = \phi (2 \cdot 5)  \\
                       & = (2-1) \cdot (5-1) \\
                       & = 1 \cdot 4         \\
              \phi(10) & = 4
          \end{align*}
    \item Portanto, temos que $7^{4} \equiv 1 \pmod {10}$. Agora, podemos simplificar a potência para calcular $7^{100}$:
          \begin{align*}
              7^{4}        & \equiv 1 \pmod {10}      \\
              (7^{4})^{25} & \equiv 1^{25} \pmod {10} \\
              7^{100}      & \equiv 1 \pmod {10}
          \end{align*}
\end{itemize}

\section{APLICAÇÃO 2 - Criptografia soma (Cifra de César)}
A Cifra de César é uma das técnicas de criptografia mais antigas que existem. Foi criada por Júlio César para fins militares.\\\\
Ela consiste na substituição de cada letra da mensagem original por uma letra $n$ posições adiante ou atrás no alfabeto, onde $n$ é a chave da criptografia.\\
O deslocamento das letras é cíclico, ou seja, ao chegar na letra \textbf{Z}, a próxima letra será a letra \textbf{A}.

\subsection{Codificação}
Matematicamente, para realizar a codificação de um texto qualquer utilizando a Cifra de César, é necessário realizar a seguinte operação para cada letra da mensagem:

\[
    C \equiv M + k \pmod{26}
\]\\
Onde $C$ é a letra da mensagem codificada, $M$ é a letra da mensagem original e $k$ é a chave de deslocamento.\\\\
\textbf{Importante:} Para realizar a conversão de uma letra para um número, deve-se seguir os índices da seguinte tabela:

\begin{center}
    \begin{tabular}{|c|c|c|c|c|c|c|c|c|c|c|c|c|c|}
        \hline
        \textbf{Letra}  & \textbf{A} & \textbf{B} & \textbf{C} & \textbf{D} & \textbf{E} & \textbf{F} & \textbf{G} & \textbf{H} & \textbf{I} & \textbf{J} & \textbf{K} & \textbf{L} & \textbf{M} \\
        \hline
        \textbf{Índice} & 0          & 1          & 2          & 3          & 4          & 5          & 6          & 7          & 8          & 9          & 10         & 11         & 12         \\
        \hline
    \end{tabular}
\end{center}
\begin{center}
    \begin{tabular}{|c|c|c|c|c|c|c|c|c|c|c|c|c|c|}
        \hline
        \textbf{Letra}  & \textbf{N} & \textbf{O} & \textbf{P} & \textbf{Q} & \textbf{R} & \textbf{S} & \textbf{T} & \textbf{U} & \textbf{V} & \textbf{W} & \textbf{X} & \textbf{Y} & \textbf{Z} \\
        \hline
        \textbf{Índice} & 13         & 14         & 15         & 16         & 17         & 18         & 19         & 20         & 21         & 22         & 23         & 24         & 25         \\
        \hline
    \end{tabular}
\end{center}

\subsection{Decodificação}
Para achar a chave de decodificação $d$, é necessário realizar a seguinte operação:
\[
    d = -(k) \bmod{26}
\]\\
Usando essa chave, o cálculo para realizar a decodificação é o seguinte:
\[
    M \equiv C + d \pmod {26}
\]\\
Depois de realizar a decodificação, cada número que representa uma letra é convertido de volta para uma letra, usando os índices da tabela apresentada na subseção anterior.

\subsection*{Exemplos}
\textbf{1.} Codificar a mensagem \textbf{ATTACK} usando a Cifra de César, com a chave $k = 8$.

\begin{itemize}
    \item Usando a tabela com os índices das letras, a mensagem ATTACK vira a seguinte mensagem numérica:\\
          \textbf{0 19 19 0 2 10}
    \item Agora é necessário realizar a operação de congruência para achar o número equivalente ao índice de cada letra:
          \begin{align*}
              C_1 & \equiv 0 + 8 \pmod {26}   \\
              C_1 & \equiv 8 \pmod {26}       \\
              \\
              C_2 & \equiv 19 + 8 \pmod {26}  \\
              C_2 & \equiv 27 \pmod {26}      \\
              C_2 & \equiv 1 \pmod {26}       \\
              \\
              C_3 & \equiv 19 + 8 \pmod {26}  \\
              C_3 & \equiv 27 \pmod {26}      \\
              C_3 & \equiv 1 \pmod {26}       \\
              \\
              C_4 & \equiv 0 + 8 \pmod {26}   \\
              C_4 & \equiv 8 \pmod {26}       \\
              \\
              C_5 & \equiv 2 + 8 \pmod {26}   \\
              C_5 & \equiv 10 \pmod {26}      \\
              \\
              C_6 & \equiv 10 + 8  \pmod {26} \\
              C_6 & \equiv 18 \pmod {26}
          \end{align*}
    \item Temos agora a mensagem numérica \textbf{8 1 1 8 10 18}.
    \item Agora, é preciso novamente usar a tabela para converter os índices de volta para letras:
    \item \textbf{Mensagem codificada:} IBBIKS
\end{itemize}
\vspace{24pt}
\textbf{2.} Decodificar a mensagem \textbf{JQG}, sabendo que a chave de codificação $k$ é igual a 4.

\begin{itemize}
    \item Para calcular a chave de decodificação $d$, temos que realizar a operação:
          \begin{align*}
              d & = -4 \bmod 26 \\
              d & = 22
          \end{align*}
    \item Passando a mensagem original para a forma numérica, temos: \textbf{9 16 6}.
    \item Realizando as operações módulo:
          \begin{align*}
              M_1 & \equiv 9 + 22 \pmod {26}  \\
              M_1 & \equiv 31 \pmod {26}      \\
              M_1 & \equiv 5 \pmod {26}       \\
              \\
              M_2 & \equiv 16 + 22 \pmod {26} \\
              M_2 & \equiv 38 \pmod {26}      \\
              M_2 & \equiv 12 \pmod {26}      \\
              \\
              M_3 & \equiv 6 + 22 \pmod {26}  \\
              M_3 & \equiv 28 \pmod{26}       \\
              M_3 & \equiv 2
          \end{align*}
    \item Agora, com a mensagem numérica \textbf{5 12 2}, podemos convertê-la para uma mensagem de texto usando a tabela.
    \item \textbf{Mensagem decodificada:} FMC
\end{itemize}

\section{APLICAÇÃO 3 - Criptografia produto}
A criptografia produto segue a mesma lógica da criptografia soma / Cifra de César, mas ao invés de somar a chave, é necessário multiplicá-la pela mensagem original para chegar na mensagem codificada.

\subsection{Codificação}
A fórmula para chegar na mensagem codificada é a seguinte:\\
\[
    C \equiv M \cdot k \pmod {n}
\]\\
Onde $C$ é a mensagem codificada, $M$ é a mensagem original, $k$ é a chave e $n$ é o módulo.\\\\
\textbf{Obs:} Para que a decodificação funcione, é necessário que $k$ e $n$ sejam coprimos, isto é, $mdc(k,n) = 1$.

\subsection{Decodificação}
Para realizar a decodificação, assim como na Cifra de César, é preciso achar a chave de decodificação $d$, que é o inverso multiplicativo de $k \bmod n$:
\[
    k \cdot d \equiv 1 \pmod {n}
\]
Depois de achar a chave de decodificação, o procedimento é parecido com a codificação:
\[
    M \equiv C \cdot d \pmod {n}
\]

\subsection*{Exemplos}
\textbf{1.} Codificar a mensagem 7, com $k = 11$ e $n = 28$.

\begin{align*}
    C & \equiv 7 \cdot 11 \pmod {28} \\
    C & \equiv 77 \pmod {28}
\end{align*}
\[
    \boxed{C \equiv 21 \pmod {28}}
\]\\\\
\vspace{24pt}
\textbf{2.} Decodificar a mensagem 12 com $k = 11$ e $n = 28$.
\begin{itemize}
    \item Precisamos calcular $d$:\\
          \[
              11 \cdot d \equiv 1 \pmod {28}
          \]
          \begin{align*}
              28 & = 11 \cdot 2 + 6                                               \\
              11 & = 6 \cdot 1 + 5                                                \\
              6  & = 5 \cdot 1 + 1                                                \\
              \\
              1  & = 6 - 1 \cdot 5                                                \\
              1  & = 6 - 1 \cdot (11 - 1 \cdot 6)                                 \\
              1  & = (28 - 2 \cdot 11) - 1 \cdot (11 - 1 \cdot (28 - 2 \cdot 11)) \\
              1  & = 28 - 2 \cdot 11 - 1 \cdot 11 + 1 \cdot 28 -2 \cdot 11        \\
              1  & = 2 \cdot 28 - 5 \cdot 11
          \end{align*}
          \[
              -5 \bmod 28  = 23
          \]
          Portanto, $d = 23$.
    \item Agora, basta realizar o cálculo de decodificação:
          \begin{align*}
              M & \equiv 12 \cdot 23 \pmod {28} \\
              M & \equiv 276 \pmod {28}         \\
              M & \equiv 24
          \end{align*}
\end{itemize}

\section{APLICAÇÃO 4 - Criptografia expoente}
A criptografia expoente segue a mesma lógica das duas últimas, mas ao invés de somar ou multiplicar a chave, é preciso tratar ela como o expoente da mensagem original.

\subsection{Codificação}
A fórmula para chegar na mensagem codificada é a seguinte:\\
\[
    C \equiv M^{k} \pmod {n}
\]\\
Onde $C$ é a mensagem codificada, $M$ é a mensagem original, $k$ é a chave e $n$ é o módulo.\\\\
\textbf{Obs:} Para que a decodificação funcione, é necessário que $k$ e $\varphi(n)$ sejam coprimos, isto é, $mdc(k,\varphi(n)) = 1$.

\subsection{Decodificação}
Para realizar a decodificação, assim como na Cifra de César, é preciso achar a chave de decodificação $d$, que é o inverso multiplicativo de $k \bmod \varphi(n)$:
\[
    k \cdot d \equiv 1 \pmod {\varphi(n)}
\]
Depois de achar a chave de decodificação, o procedimento é parecido com a codificação:
\[
    M \equiv C^{d} \pmod {n}
\]

\subsection*{Exemplos}
\textbf{1.} Codificar 12, com $k=3$ e $n=23$.

\begin{align*}
    C & \equiv 12^{3} \pmod {23}          \\
    C & \equiv 12 \cdot 12^{2} \pmod {23} \\
    C & \equiv 12 \cdot 144 \pmod {23}    \\
    C & \equiv 12 \cdot 6 \pmod {23}      \\
    C & \equiv 72 \pmod {23}
\end{align*}
\[
    \boxed{C \equiv 3 \pmod {23}}
\]

\vspace{24pt}

\section{APLICAÇÃO 5 - Criptografia RSA}
O método de criptografia RSA é um criptossistema \textbf{assimétrico}, ou seja, o remetente e o destinatário não precisam compartilhar a mesma chave para codificar e decodificar a mensagem. O sistema se baseia na dificuldade de fatoração de números primos grandes.\\\\
O método consiste nos seguintes processos:

\subsection{Geração das chaves}
Definem-se dois números primos $\mathbf{p}$ e $\mathbf{q}$. Na prática, esses primos são muitos grandes (com mais de 200 dígitos), mas aqui eles receberão valores pequenos, para fins demonstrativos.\\\\
Depois, multiplica-se $p$ e $q$ e atribui-se o resultado a $\mathbf{n}$:
\[
    n = p \cdot q
\]
Escolhe-se um número $k$ tal que $k$ e $\varphi(n)$ são coprimos, ou seja:
\[
    mdc(k, \varphi(n)) = 1
\]
Para gerar a chave de decodificação $\mathbf{d}$, calcula-se o inverso multiplicativo de $k \bmod \varphi(n)$:
\[
    k \cdot d \equiv 1 \pmod{\varphi(n)}
\]
Com todos esses dados, é possível prosseguir para o próximo passo.

\vspace{24pt}
\subsubsection{Chaves públicas}
As chaves que podem ser disponibilizadas publicamente são $(k,n)$.

\subsubsection{Chaves privadas}
As chaves que são disponibilizadas apenas para quem vai realizar a decodificação são $(d,n)$.

\subsection{Codificação}
Para realizar a codificação, é realizada a seguinte operação:
\[
    m^{k} \equiv c \pmod {n}
\]
Onde:
\begin{itemize}
    \item $m$: mensagem original
    \item $c$: mensagem codificada
\end{itemize}

\subsection{Decodificação}
Para realizar a decodificação, é realizada a seguinte operação:
\[
    c^{d} \equiv m \pmod {n}
\]
Onde:
\begin{itemize}
    \item $c$: mensagem codificada
    \item $m$: mensagem original
\end{itemize}

\subsection*{Exemplos}
\textbf{1.} Codificar o número 7, com $p = 3$, $q = 11$ e $k = 3$.

\begin{itemize}
    \item Primeiro, é necessário calcular $n$.
          \[
              n = 3 \cdot 11 = 33
          \]
    \item Agora, é preciso verificar se $mdc(k, \varphi(n)) = 1$.
          \begin{align*}
              \varphi(n) = \varphi(33) & = \varphi(11 \cdot 3)    \\
                                       & = (11 - 1) \cdot (3 - 1) \\
                                       & = 10 \cdot 2             \\
              \varphi(33)              & = 20
              \\
              \\
              mdc(3, 20)               & = 1
          \end{align*}
          Como o resultado é igual a 1, é possível realizar a criptografia.
    \item Por último, realizamos o cálculo para encontrar o número codificado equivalente a 7
          \begin{align*}
              m^{k}         & \equiv c \pmod {n}  \\
              7^{3}         & \equiv c \pmod {33} \\
              7^{2} \cdot 7 & \equiv c \pmod {33} \\
              49 \cdot 7    & \equiv c \pmod {33} \\
              16 \cdot 7    & \equiv c \pmod {33} \\
              112           & \equiv c \pmod {33}
          \end{align*}
          \[
              \boxed{
                  13 \equiv c \pmod {33}
              }
          \]
\end{itemize}
\vspace{24pt}
\textbf{2.} Usando os dados do exemplo anterior, decodificar o número 28.

\begin{align*}
    k \cdot d & \equiv 1 \pmod {\varphi(n)} \\
    3 \cdot d & \equiv 1 \pmod {20}
\end{align*}
\[
    \boxed{d = 7}
\]

\begin{align*}
    c^{d}                              & \equiv m \pmod {n}  \\
    28^{7}                             & \equiv m \pmod {33} \\
    (28^{2})^{2} \cdot 28^{2} \cdot 28 & \equiv m \pmod {33} \\
    784^{2} \cdot 784 \cdot 28         & \equiv m \pmod {33} \\
    25^{2} \cdot 25 \cdot 28           & \equiv m \pmod {33} \\
    625 \cdot 700                      & \equiv m \pmod {33} \\
    31 \cdot 7                         & \equiv m \pmod {33} \\
    217                                & \equiv m \pmod {33}
\end{align*}
\[
    \boxed{19 \equiv m \pmod {33}}
\]

\end{document}