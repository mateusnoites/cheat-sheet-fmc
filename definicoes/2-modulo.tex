\section{DEFINIÇÃO 2 - Módulo}
Sejam $a, b, m \in \mathbb{Z}$. $a$ é congruente a $b$ módulo $m$ se, e somente se $m \mid a-b$.
\[
    a \equiv b \pmod m \iff m \mid a-b
\]
Também é possível usar o módulo para representar o resto de uma divisão. Pela definição de divisão euclidiana, sabe-se que um número arbitrário $D \in \mathbb{Z}$ pode ser representado como
\[
    D = dq+r\text{.}
\]
Com $\mathbf{D}$ sendo o \textbf{dividendo}, $\mathbf{d}$ o \textbf{divisor}, $\mathbf{q}$ o \textbf{quociente} e $\mathbf{r}$ o \textbf{resto} ($0 \leq r < |d|$).\\
Nesse sentido, podemos afirmar que:
\[
    D \bmod d = r\text{.}
\]