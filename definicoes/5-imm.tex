\section{DEFINIÇÃO 5 - Inverso Multiplicativo Modular}
Este é um conceito essencial que se relaciona com o conceito de congruência linear (\textbf{definição 6}).\\\\
Sejam $a,m \in \mathbb{Z}$. O inverso multiplicativo modular de $a \bmod m$ é o inteiro $x$ tal que:
\[
    ax \equiv 1 \pmod m
\]

\subsection{Condição de existência}
O inverso multiplicativo modular de $a \bmod m$ existe se, e somente se $mdc(a,m) = 1$, isto é, se $a$ e $m$ forem \textbf{coprimos} ou \textbf{primos entre si}.\\

\subsection{Métodos para encontrar}
É possível encontrar o inverso multiplicativo de $a \bmod m$ facilmente usando o \textbf{teorema 2 - teorema de Bézout}.
\\\\Ao escrever o $mdc(a,m)$ como uma combinação linear de $a$ e $m$, o coeficiente de $a$ é o seu inverso multiplicativo.

\subsection*{Exemplos}
\textbf{1.} Encontrar o inverso multiplicativo de $3 \bmod 7$.

\begin{align*}
     & 3x \equiv 1 \pmod 7                                             \\
    \\
     & \text{Primeiro, precisamos calcular $mdc(3,7)$.}                \\
    \\
     & 7 = 3 \cdot 2 + 1                                               \\
     & 3 = 1 \cdot 3 + 0                                               \\
    \\
     & \text{Como $mdc(3,7) = 1$, o inverso multiplicativo existe.}    \\
     & \text{Agora, escrevemos 1 como uma combinação linear de 3 e 7.} \\
    \\
     & 1 = 1 \cdot 7 - 2 \cdot 3                                       \\
    \\
     & \text{O coeficiente de 3 é -2, então $x = -2$.}
\end{align*}

Como o inverso multiplicativo encontrado é um número negativo, podemos fazer a operação $x \bmod m$ para encontrar um inverso multiplicativo positivo (o que é uma boa prática).

\[
    -2 \bmod 7 = 5
\]

Logo, o inverso multiplicativo que procuramos é $\mathbf{5}$.\\

\textbf{Obs:} Para encontrar um inverso multiplicativo positivo também é possível somar $m$ a $x$ até que $x$ seja maior ou igual a 1.