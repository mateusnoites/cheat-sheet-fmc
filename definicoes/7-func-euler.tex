\section{DEFINIÇÃO 7 - Função de Euler}
A função de Euler, função totiente ou função phi é fundamental para os próximos tópicos, especialmente o de criptografia de chave pública ou assimétrica.
\begin{gather*}
    \text{Notação:}\\
    \varphi(n)
\end{gather*}
Ela calcula a quantidade de números naturais não nulos $m$ menores ou iguais a $n$ tais que $m$ e $n$ são relativamente primos, ou seja, $mdc(m,n) = 1$

\subsection*{Exemplos}
\textbf{1.} Calcular $\varphi(6)$:

\begin{itemize}
    \item Os números $m \leq 6$ são $\{1,2,3,4,5,6\}$.
          \begin{itemize}
              \item $mdc(1,6) = 1$ \textbf{(coprimo)}
              \item $mdc(2,6) = 2$ (não coprimo)
              \item $mdc(3,6) = 3$ (não coprimo)
              \item $mdc(4,6) = 2$ (não coprimo)
              \item $mdc(5,6) = 1$ \textbf{(coprimo)}
              \item $mdc(6,6) = 6$ (não coprimo)
          \end{itemize}
    \item Os coprimos são $\mathbf{1}$ e $\mathbf{5}$, então $\varphi(6) = 2$.
\end{itemize}
\vspace{12pt}
\textbf{2.} Calcular $\varphi(7)$:

\begin{itemize}
    \item Como 7 é um número primo, todos os números naturais não nulos até $7-1$ são coprimos com ele, portanto $\varphi(7)=6$.
\end{itemize}

\subsection{Principais fórmulas}
Existem algumas fórmulas e propriedades que ajudam a calcular $\varphi(n)$. Sejam $p,q,m,n \in \mathbb{Z}$ com $p \neq q$, $p$ e $q$ números primos e $m$ e $n$ coprimos.

\setcounter{equation}{0}
\begin{align}
     & \varphi(p) = p-1                                 \\
     & \varphi(p \cdot q) = (p-1)\cdot(q-1)             \\
     & \varphi(p^{2}) = p \cdot (p-1)                   \\
     & \varphi(m \cdot n) = \varphi(m) \cdot \varphi(n)
\end{align}

\subsection*{Exemplos}
\textbf{3.} Calcular $\varphi(19)$:

\begin{itemize}
    \item Pela fórmula \textbf{(1)}, $\varphi(19) = 19 - 1 = 18$.
\end{itemize}
\vspace{12pt}
\textbf{4.} Calcular $\varphi(35)$:

\begin{itemize}
    \item Decompondo 35 em fatores primos, obtemos $35 = 7 \cdot 5$.
    \item Pela fórmula \textbf{(2)}, $\varphi(35) = \varphi(7 \cdot 5) = (7 - 1) \cdot (5 - 1) = 6 \cdot 4 = 24$.
\end{itemize}
\vspace{12pt}
\textbf{5.} Calcular $\varphi(49)$:

\begin{itemize}
    \item Sabe-se que $49 = 7^{2}$.
    \item Pela fórmula \textbf{(3)}, $\varphi(49) = \varphi(7^{2}) = 7 \cdot (7 - 1) = 7 \cdot 6 = 42$.
\end{itemize}
