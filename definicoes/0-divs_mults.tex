\section*{DEFINIÇÃO 0 - Divisores e Múltiplos}
Essa definição não faz, realmente, parte do conteúdo, mas é fundamental para o entendimento de todas as próximas definições.
\subsection*{Parte 1 - Divisores}
Um número inteiro $d$ é divisor de um número inteiro $a$ se, e somente se, ao dividir $a$ por $d$, o resto for \textbf{zero}, ou seja, a divisão é exata.\\
Por exemplo, para o número $24$ temos $8$ divisores. São eles:\\
\[
    D = \{1,2,3,4,6,8,12,24\}
\]
\subsection*{Parte 2 - Múltiplos}
Um número inteiro $b$ é múltiplo de um número inteiro $a$ se, e somente se, existe um número inteiro $k$ tal que:
\[
    b = ak
\]
Por exemplo, se $a = 3$, os múltiplos de 3 são:
\begin{itemize}
    \item Se $k = 2$, $b = 3 \cdot 2 = 6$
    \item Se $k = 5$, $b = 3 \cdot 5 = 15$
\end{itemize}
O conjunto dos múltiplos de $3$ é:
\[
    M(3) = \{\dots, -9, -6, -3, 0, 3, 6, 9, 12, \dots\}
\]
\subsection*{Parte 3 - Relação entre Divisores e Múltiplos}
Se $b$ é um \textbf{múltiplo} de $a$, isso significa que $a$ é um \textbf{divisor} de $b$. Esta relação será explorada melhor na definição de \textbf{divisibilidade}.