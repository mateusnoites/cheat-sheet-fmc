\section{DEFINIÇÃO 6 - Congruência Linear}
Uma congruência linear é uma equação na forma $ax \equiv b \pmod m$. Uma congruência linear tem solução se, e somente se $mdc(a,m) \mid b$.\\
\\
Para resolver a congruência linear, é necessário seguir os seguintes passos:

\begin{enumerate}

    \item Encontrar o inverso multiplicativo de $a$ -- denotado por $\overline{a}$ ou $a^{-1}$ -- utilizando o método descrito na \textbf{definição 5}.
          \[
              a \overline{a} \equiv 1 \pmod m
          \]

    \item Multiplicar os dois lados da congruência por $\overline{a}$.
          \[
              \overline{a}ax \equiv \overline{a}b \pmod m
          \]
    \item Simplificando, o resultado fica:
          \[
              x \equiv \overline{a}b \pmod m
          \]
    \item Se $\overline{a}b < 1$ ou $\overline{a}b \geq m$, é necessário executar a operação $(\overline{a}b \bmod m)$ para encontrar a menor congruência natural.

\end{enumerate}

\subsection*{Exemplos}
\textbf{1.} Calcular $17x \equiv 82 \pmod{11}$

\begin{align*}
     & 17\overline{a} \equiv 1 \pmod{11}                                \\
    \\
     & 17 = 11 \cdot 1 + 6                                              \\
     & 11 = 6 \cdot 1 + 5                                               \\
     & 6 = 5 \cdot 1 + 1                                                \\
    \\
     & 1 = 6 - 1 \cdot 5                                                \\
     & 1 = 6 - 1 \cdot (11 - 1 \cdot 6)                                 \\
     & 1 = (17 - 1 \cdot 11) - 1 \cdot (11 - 1 \cdot (17 - 1 \cdot 11)) \\
     & 1 = 17 - 1 \cdot 11 - 1 \cdot 11 + 1 \cdot 17 -1 \cdot 11        \\
     & 1 = 2 \cdot 17 - 3 \cdot 11                                      \\
    \\
     & \overline{a} = 2                                                 \\
    \\
     & 2 \cdot 17x \equiv 2 \cdot 82 \pmod {11}                         \\
     & 34x \equiv 164 \pmod {11}                                        \\
     & \boxed{x \equiv 10 \pmod {11}}                                   \\
\end{align*}

Também é possível escrever a solução na forma de um conjunto solução, usando as definições 2 e 1 \textbf{(módulo e divisibilidade)}, respectivamente:
\begin{align*}
    11  & \mid x - 10                                                \\
    11k & = x - 10                                                   \\
    x   & = 11k + 10                                                 \\
    \\
    S   & = \{x \in \mathbb{Z} \mid x = 10 + 11k, k \in \mathbb{Z}\}
\end{align*}