\section{APLICAÇÃO 4 - Criptografia expoente}
A criptografia expoente segue a mesma lógica das duas últimas, mas ao invés de somar ou multiplicar a chave, é preciso tratar ela como o expoente da mensagem original.

\subsection{Codificação}
A fórmula para chegar na mensagem codificada é a seguinte:\\
\[
    C \equiv M^{k} \pmod {n}
\]\\
Onde $C$ é a mensagem codificada, $M$ é a mensagem original, $k$ é a chave e $n$ é o módulo.\\\\
\textbf{Obs:} Para que a decodificação funcione, é necessário que $k$ e $\varphi(n)$ sejam coprimos, isto é, $mdc(k,\varphi(n)) = 1$.

\subsection{Decodificação}
Para realizar a decodificação, assim como na Cifra de César, é preciso achar a chave de decodificação $d$, que é o inverso multiplicativo de $k \bmod \varphi(n)$:
\[
    k \cdot d \equiv 1 \pmod {\varphi(n)}
\]
Depois de achar a chave de decodificação, o procedimento é parecido com a codificação:
\[
    M \equiv C^{d} \pmod {n}
\]

\subsection*{Exemplos}
\textbf{1.} Codificar 12, com $k=3$ e $n=23$.

\begin{align*}
    C & \equiv 12^{3} \pmod {23}          \\
    C & \equiv 12 \cdot 12^{2} \pmod {23} \\
    C & \equiv 12 \cdot 144 \pmod {23}    \\
    C & \equiv 12 \cdot 6 \pmod {23}      \\
    C & \equiv 72 \pmod {23}
\end{align*}
\[
    \boxed{C \equiv 3 \pmod {23}}
\]

\vspace{24pt}