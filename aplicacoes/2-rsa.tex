\section*{APLICAÇÃO 2 - Criptografia RSA}
O método de criptografia RSA é um criptossistema \textbf{assimétrico}, ou seja, o remetente e o destinatário não precisam compartilhar a mesma chave para codificar e decodificar a mensagem. O sistema se baseia na dificuldade de fatoração de números primos grandes.\\\\
O método consiste nos seguintes processos:

\subsection*{1. Geração das chaves}
Definem-se dois números primos $\mathbf{p}$ e $\mathbf{q}$. Na prática, esses primos são muitos grandes (com mais de 200 dígitos), mas aqui eles receberão valores pequenos, para fins demonstrativos.\\\\
Depois, multiplica-se $p$ e $q$ e atribui-se o resultado a $\mathbf{n}$:
\[
    n = p \cdot q
\]
Escolhe-se um número $k$ tal que $k$ e $\phi(n)$ são coprimos, ou seja:
\[
    mdc(k, \phi(n)) = 1
\]
Para gerar a chave de decodificação $\mathbf{d}$, calcula-se o inverso multiplicativo de $k \bmod \phi(n)$:
\[
    k \cdot d \equiv 1 \pmod{\phi(n)}
\]
Com todos esses dados, é possível prosseguir para o próximo passo.

\vspace{24pt}
\subsubsection*{Chaves públicas}
As chaves que podem ser disponibilizadas publicamente são $(k,n)$.

\subsubsection*{Chaves privadas}
As chaves que são disponibilizadas apenas para quem vai realizar a decodificação são $(d,n)$.

\subsection*{2. Codificação}
Para realizar a codificação, é realizada a seguinte operação:
\[
    m^{k} \equiv c \pmod {n}
\]
Onde:
\begin{itemize}
    \item $m$: mensagem original
    \item $c$: mensagem codificada
\end{itemize}

\subsection*{3. Decodificação}
Para realizar a decodificação, é realizada a seguinte operação:
\[
    c^{d} \equiv m \pmod {n}
\]
Onde:
\begin{itemize}
    \item $c$: mensagem codificada
    \item $m$: mensagem original
\end{itemize}

\subsection*{Exemplos}
\textbf{1.} Codificar o número 7, com $p = 3$, $q = 11$ e $k = 3$.

\begin{itemize}
    \item Primeiro, é necessário calcular $n$.
          \[
              n = 3 \cdot 11 = 33
          \]
    \item Agora, é preciso verificar se $mdc(k, \phi(n)) = 1$.
          \begin{align*}
              \phi(n) = \phi(33) & = \phi(11 \cdot 3)       \\
                                 & = (11 - 1) \cdot (3 - 1) \\
                                 & = 10 \cdot 2             \\
              \phi(33)           & = 20
              \\
              \\
              mdc(3, 20)         & = 1
          \end{align*}
          Como o resultado é igual a 1, é possível realizar a criptografia.
    \item Por último, realizamos o cálculo para encontrar o número codificado equivalente a 7
          \begin{align*}
              m^{k}         & \equiv c \pmod {n}  \\
              7^{3}         & \equiv c \pmod {33} \\
              7^{2} \cdot 7 & \equiv c \pmod {33} \\
              49 \cdot 7    & \equiv c \pmod {33} \\
              16 \cdot 7    & \equiv c \pmod {33} \\
              112           & \equiv c \pmod {33}
          \end{align*}
          \[
              \boxed{
                  13 \equiv c \pmod {33}
              }
          \]
\end{itemize}
\vspace{24pt}
\textbf{2.} Usando os dados do exemplo anterior, decodificar o número 28.

\begin{align*}
    k \cdot d & \equiv 1 \pmod {\phi(n)} \\
    3 \cdot d & \equiv 1 \pmod {20}
\end{align*}
\[
    \boxed{d = 7}
\]

\begin{align*}
    c^{d}                              & \equiv m \pmod {n}  \\
    28^{7}                             & \equiv m \pmod {33} \\
    (28^{2})^{2} \cdot 28^{2} \cdot 28 & \equiv m \pmod {33} \\
    784^{2} \cdot 784 \cdot 28         & \equiv m \pmod {33} \\
    25^{2} \cdot 25 \cdot 28           & \equiv m \pmod {33} \\
    625 \cdot 700                      & \equiv m \pmod {33} \\
    31 \cdot 7                         & \equiv m \pmod {33} \\
    217                                & \equiv m \pmod {33}
\end{align*}
\[
    \boxed{19 \equiv m \pmod {33}}
\]