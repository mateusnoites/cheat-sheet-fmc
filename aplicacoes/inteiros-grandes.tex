\section*{APLICAÇÃO 1 - Números Inteiros Grandes}
Na área da computação, muitas vezes é necessário computar números inteiros grandes que, a princípio, não podem ser computados por um processador comum de computador.\\

\subsection*{Codificação}
Sejam $m_1, m_2, \dots, m_n$ inteiros maiores que 1 e primos entre si, com $m$ sendo o produto entre eles e  $a \in \mathbb{Z}$ tal que $0 \leq a < m$. É possível representar todos os números $a$ como a $n$-upla:
\[
    a = (a \bmod m_1, a \bmod m_2, \dots, a \bmod m_n).
\]
\\
Por exemplo, se definirmos $m_1 = 3$, $m_2 = 5$, teremos as seguintes representações:

\begin{align*}
     & 0 = (0, 0) & \qquad & 5 = (2, 0) & \qquad & 10 = (1, 0) \\
     & 1 = (1, 1) & \qquad & 6 = (0, 1) & \qquad & 11 = (2, 1) \\
     & 2 = (2, 2) & \qquad & 7 = (1, 2) & \qquad & 12 = (0, 2) \\
     & 3 = (0, 3) & \qquad & 8 = (2, 3) & \qquad & 13 = (1, 3) \\
     & 4 = (1, 4) & \qquad & 9 = (0, 4) & \qquad & 14 = (2, 4) \\
\end{align*}

\subsection*{Decodificação}
Dada uma $n$-upla e seus $m_i$, é possível chegar facilmente ao valor representado usando o \textbf{teorema chinês do resto} (teorema 3).\\

Como exemplo, podemos tentar descobrir o valor representado por uma dupla aleatória do exemplo de \textbf{codificação}. Vamos usar a dupla $\mathbf{(2,1)}$ e $m_1 = 3, m_2 = 5$:

\begin{align*}
     & x = (2,1) = (x \bmod 3, x \bmod 5)                            \\
    \\
     & \begin{cases}
           x \equiv 2 \pmod 3 \\
           x \equiv 1 \pmod 5 \\
       \end{cases}                                            \\
    \\
     & M = 3 \cdot 5 = 15                                            \\
    \\
     & M_1 = 5                                                       \\
     & M_2 = 3                                                       \\
    \\
     & y_1 = 2                                                       \\
     & y_2 = 2                                                       \\
    \\
     & x \equiv (2 \cdot 5 \cdot 2) + (1 \cdot 3 \cdot 2) \pmod {15} \\
     & x \equiv 20 + 6 \pmod {15}                                    \\
     & x \equiv 26 \pmod {15}
\end{align*}
\[
    \boxed{x \equiv 11 \pmod {15}}
\]\\
\textbf{Obs:} Nesse caso, podemos considerar como verdadeira apenas a primeira equivalência, portanto $x = 11$.

\subsection*{Operações aritméticas}
Para realizar operações aritméticas com as $n$-uplas, basta realizar tal operação entre o $i$-ésimo termo da primeira $n$-upla com o seu respectivo na segunda $n$-upla, e com o resultado realizar a operação módulo com o $m_i$ correspondente.

\subsubsection*{Restrições}
Para realizar a operação, o valor resultante deve poder ser escrito também como uma $n$-upla. Portanto, o resultado deve ser um dos possíveis valores de $a$ ($0 \leq a < m$).

\subsubsection*{Exemplo}
Sejam $a = 2, b = 3, m_1 = 3, m_2 = 5$.
\begin{align*}
    a + b & = 2 + 3 = (2,2) + (0,3)              \\
    a + b & = ((2 + 0) \bmod 3, (2 + 3) \bmod 5) \\
    a + b & = (2 \bmod 3, 5 \bmod 5)             \\
    a + b & = (2, 0)                             \\
    a + b & = 5
\end{align*}

\begin{align*}
    a \cdot b & = 2 \cdot 3 = (2,2) \cdot (0,3)              \\
    a \cdot b & = ((2 \cdot 0) \bmod 3, (2 \cdot 3) \bmod 5) \\
    a \cdot b & = (0 \bmod 3, 6 \bmod 5)                     \\
    a \cdot b & = (0, 1)                                     \\
    a \cdot b & = 6
\end{align*}