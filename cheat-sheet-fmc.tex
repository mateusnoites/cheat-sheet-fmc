\documentclass[a4paper,12pt]{article}
\usepackage{amsmath}
\usepackage{amssymb}
\usepackage{xcolor}
\usepackage{calc}
\usepackage[left=3cm,top=3cm,bottom=2cm,right=2cm]{geometry}

\title{\textbf{Cheat Sheet: FMC - Unidade 2}}
\author{Mateus Dias \\ Tecnologia da Informação - IMD/UFRN}
\date{11/10/2025}

\begin{document}

\maketitle
\vspace{12pt}

\section*{DEFINIÇÃO 0 - Divisores e Múltiplos}
Essa definição não faz, realmente, parte do conteúdo, mas é fundamental para o entendimento de todas as próximas definições.
\subsection*{Parte 1 - Divisores}
Um número inteiro $d$ é divisor de um número inteiro $a$ se, e somente se, ao dividir $a$ por $d$, o resto for \textbf{zero}, ou seja, a divisão é exata.\\
Por exemplo, para o número $24$ temos $8$ divisores. São eles:\\
\[
    D = \{1,2,3,4,6,8,12,24\}
\]
\subsection*{Parte 2 - Múltiplos}
Um número inteiro $b$ é múltiplo de um número inteiro $a$ se, e somente se, existe um número inteiro $k$ tal que:
\[
    b = ak
\]
Por exemplo, se $a = 3$, os múltiplos de 3 são:
\begin{itemize}
    \item Se $k = 2$, $b = 3 \cdot 2 = 6$
    \item Se $k = 5$, $b = 3 \cdot 5 = 15$
\end{itemize}
O conjunto dos múltiplos de $3$ é:
\[
    M(3) = \{\dots, -9, -6, -3, 0, 3, 6, 9, 12, \dots\}
\]
\subsection*{Parte 3 - Relação entre Divisores e Múltiplos}
Se $b$ é um \textbf{múltiplo} de $a$, isso significa que $a$ é um \textbf{divisor} de $b$. Esta relação será explorada melhor na definição de \textbf{divisibilidade}.
\vspace{6pt}\\

\section*{DEFINIÇÃO 1 - Divisibilidade}
Sejam $a, b \in \mathbb{Z}$. Dizemos que $a$ divide $b$ se, e somente se $\exists k \in \mathbb{Z}$ tal que: $ak = b$.
\[
    a \mid b \iff (\exists k \in \mathbb{Z}) (ak = b).
\]
\vspace{-5pt}\\

\section{DEFINIÇÃO 2 - Módulo}
Sejam $a, b, m \in \mathbb{Z}$. $a$ é congruente a $b$ módulo $m$ se, e somente se $m \mid a-b$.
\[
    a \equiv b \pmod m \iff m \mid a-b
\]
Também é possível usar o módulo para representar o resto de uma divisão. Pela definição de divisão euclidiana, sabe-se que um número arbitrário $D \in \mathbb{Z}$ pode ser representado como
\[
    D = dq+r\text{.}
\]
Com $\mathbf{D}$ sendo o \textbf{dividendo}, $\mathbf{d}$ o \textbf{divisor}, $\mathbf{q}$ o \textbf{quociente} e $\mathbf{r}$ o \textbf{resto} ($0 \leq r < |d|$).\\
Nesse sentido, podemos afirmar que:
\[
    D \bmod d = r\text{.}
\]
\vspace{6pt}\\

\section{TEOREMA 1}
Sejam $a,b,m \in \mathbb{Z}$, com $m > 0$. $a$ é congruente a $b$ módulo $m$ se, e somente se $a \bmod m = b \bmod m$.
\[
      a \equiv b \pmod m \iff a \bmod m = b \bmod m \text{.}
\]
\subsection*{Prova ($\implies$)}
Suponha $a, b, m \in \mathbb{Z}$ tal que $a \equiv b \pmod m$.\\
Pelas definições 2 e 1, respectivamente, temos que:
\begin{flalign*}
       & m \mid a -b                          \\
       & mk = a-b                             \\
       & a = b + mk \quad \text{\textbf{(I)}}
\end{flalign*}
Pela definição do resto, ao dividir $b$ por $m$, temos:
\[
      b = q_b \cdot m + r_b \quad \text{\textbf{(II)}}
\]
Onde $q_b \in \mathbb{Z}$ e $r_b = b \bmod m$, com $0 \leq r_b < m$.\\
Substituindo \textbf{(II)} em \textbf{(I)}:
\begin{flalign*}
       & a = (q_b \cdot m + r_b) + mk \\
       & a = mq_b + mk + r_b          \\
       & a = m\cdot(q_b + k) + r_b
\end{flalign*}
Tome $q_a = q_b + k$. Como $q_a \in \mathbb{Z}$ e $0 \leq r_b < m$, podemos dizer que $r_b$ é o resto da divisão de $a$ por $m$, isto é, $a \bmod m = r_b$.\\
Como $r_b = b \bmod m$, temos que:
\[
      a \bmod m = b \bmod m \text{.}
\]
\subsection*{Prova ($\impliedby$)}
Suponha $a, b, m \in \mathbb{Z}$ tal que $a \bmod m = b \bmod m$.
Seja $r$ o valor comum do resto, de forma que:

\[
      r = a\bmod m = b \bmod m
\]

Pela definição do resto, podemos escrever $a$ e $b$ como:
\begin{flalign*}
       & a = q_a \cdot m + r \\
       & b = q_b \cdot m + r
\end{flalign*}

Onde $q_a, q_b \in \mathbb{Z}$ e $0 \leq r < m$.
Dessa forma, a diferença $a - b$ fica desta forma:
\begin{flalign*}
       & a - b = (q_a \cdot m + r) - (q_b \cdot m + r) \\
       & a - b = q_a \cdot m + r - q_b \cdot m - r     \\
       & a - b = m \cdot (q_a-q_b)
\end{flalign*}

Seja $k = q_a-q_b$. Como $q_a \in \mathbb{Z}$ e $q_b \in \mathbb{Z}$ temos que $k \in \mathbb{Z}$. Portanto:

\[
      a-b = mk
\]

Pela definição 1 e 2, respectivamente, temos que:
\begin{flalign*}
       & m \mid a-b         \\
       & a \equiv b \pmod m
\end{flalign*}

\begin{flalign*}
       &  & \blacksquare
\end{flalign*}

\section{DEFINIÇÃO 3 - Máximo Divisor Comum}
Sejam $a, b \in \mathbb{Z}$ com $a, b \neq 0$. O MDC de $a$ e $b$, denotado por $mdc(a, b)$ é o único inteiro positivo $d$ que satisfaz as seguintes condições:
\begin{enumerate}
    \item $d \mid a$
    \item $d \mid b$
    \item $\forall c \in \mathbb{Z} [((c \mid a) \land (c \mid b)) \implies c \mid d]$
\end{enumerate}

Em outros termos, $d$ é o maior número inteiro positivo que divide $a$ e $b$ ao mesmo tempo.

\subsection*{Exemplos}

\textbf{1.} Calcular o $mdc(12, 18)$.
\begin{align*}
     & \text{Divisores de 12: } \{1,2,3,4,6,12\} \\
     & \text{Divisores de 18: } \{1,2,3,6,9,18\} \\
     & \text{Divisores comuns: } \{1,2,3,6\}     \\
     & \textbf{Máximo Divisor Comum (MDC): } 6
\end{align*}
\vspace{-10pt}

\subsection{Algoritmo de Euclides}
O algoritmo de Euclides é um método simples para encontrar o MDC entre dois números inteiros diferentes de zero. Ele é um derivado da divisão euclidiana:
\[
    D = dq + r\text{.}
\]
Com $\mathbf{D}$ sendo o \textbf{dividendo}, $\mathbf{d}$ o \textbf{divisor}, $\mathbf{q}$ o \textbf{quociente} e $\mathbf{r}$ o \textbf{resto} ($0 \leq r < |d|$).\\\\
Se queremos calcular $mdc(a,b)$, podemos assumir $D_1 = \max(a,b)$ como o dividendo inicial e $d_1 = \min(a,b)$ como o divisor inicial.\\\\
O algoritmo procede em etapas sucessivas, onde o resto de cada divisão se torna o novo divisor e o divisor anterior se torna o novo dividendo, até que $r_i = 0$ (onde $i$ é o número de iterações). O último resto \textbf{não nulo} é o $mdc(a,b)$.

\subsection*{Exemplos}
\textbf{2.} Calcular o $mdc(270, 192)$.
\begin{align}
     & 270 = 192 \cdot 1 + 78 \\
     & 192 = 78 \cdot 2 + 36  \\
     & 78 = 36 \cdot 2 + 6    \\
     & 36 = 6 \cdot 6 + 0
\end{align}
Portanto, o $\mathbf{mdc(270,192)}$ é igual ao último resto não nulo, ou seja, $\mathbf{6}$.

\section*{DEFINIÇÃO 4 - Mínimo Múltiplo Comum}
Sejam $a, b \in \mathbb{Z}$. O $mmc(a,b)$ é o menor número inteiro positivo que é múltiplo de $a$ e $b$ simultaneamente.

\subsection*{Exemplos}
\textbf{1.} Calcular o $mmc(4,6)$.
\begin{align*}
    \textbf{Múltiplos de 4: } \{4,8,\mathbf{12},16,20,24,...\}  \\
    \textbf{Múltiplos de 6: } \{6,\mathbf{12},18,24,30,36,...\} \\
\end{align*}
O menor dos múltiplos comuns é $12$, portanto $mmc(4,6) = 12$.\\
\vspace{-10pt}\\

\subsection*{Métodos para calcular o MMC}
É possível conectar os conceitos de MMC e MDC com uma fórmula relacionada à Matemática Discreta:

\[
    mmc(a,b) = \frac{|a \cdot b|}{mdc(a,b)}
\]
\\
Este é o método que apresenta maior eficiência computacional para calcular o MMC entre dois números, mas também existe o método da fatoração prima (mais útil para calcular o MMC entre três ou mais números):

\begin{enumerate}
    \item \textbf{Fatore} todos os números em seus fatores primos;
    \item O \textbf{MMC} é o produto de todos os fatores primos distintos, cada um elevado à maior potência em que ele aparece em qualquer uma das fatorações.
\end{enumerate}

\subsection*{Exemplos}
\textbf{2.} Calcular o $mmc(12,18)$ usando o primeiro método.

\begin{enumerate}
    \item Calcular o \textbf{mdc(12,18)}:\\
          Segundo o método apresentado na \textbf{definição 3}:
          \begin{align}
              18 & = 12 \cdot 1 + 6 \tag{1} \\
              12 & = 6 \cdot 2 + 0 \tag{2}
          \end{align}
          Portanto, $mdc(12,18) = \mathbf{6}$.
    \item Substituir na fórmula:\\
          \begin{align*}
               & mmc(12,18) = \frac{|12 \cdot 18|}{6} \\
              \\
               & mmc(12,18) = \frac{216}{6}           \\
              \\
               & mmc(12,18) = 36
          \end{align*}
\end{enumerate}
\textbf{3.} Calcular o $mmc(12,18)$ usando o segundo método.

\begin{enumerate}
    \item Fatore 12 e 18 em seus respectivos fatores primos:
          \begin{align*}
               & 12 = 2^2 \cdot 3^1 \\
               & 18 = 2^1 \cdot 3^2
          \end{align*}
    \item Fatores e maiores potências:
          \begin{itemize}
              \item Fator 2: $2^2$
              \item Fator 3: $3^2$
          \end{itemize}
    \item Cálculo:
          \begin{align*}
               & mmc(12,18) = 2^2 \cdot 3^2 \\
               & mmc(12,18) = 4 \cdot 9     \\
               & mmc(12,18) = 36
          \end{align*}
\end{enumerate}

O MMC entre $a$ e $b$ também pode ser interpretado como ``o primeiro número em que $a$ irá \textit{se encontrar} com $b$ quando ambos forem multiplicados por números naturais''.

\section{TEOREMA 2 - Teorema de Bézout}
Sejam $a, b \in \mathbb{Z}$ com $a,b > 0$. O $mdc(a,b)$ pode ser escrito como uma combinação linear de $a$ e $b$:
\[
    mdc(a,b) = sa + tb
\]
Com $s,t \in \mathbb{Z}$.\\
O método para descobrir os valores de $s$ e $t$ é substituir consecutivamente os valores no algoritmo de Euclides.

\subsection*{Exemplos}
\textbf{1.} Expressar o mdc(270,192) como uma combinação linear de 270 e 192.
\begin{align*}
     & 270 = 192 \cdot 1 + 78                                                \\
     & 192 = 78 \cdot 2 + 36                                                 \\
     & 78 = 36 \cdot 2 + 6                                                   \\
     & 36 = 6 \cdot 6 + 0                                                    \\
    \\
     & 6 = 78 - 2 \cdot 36                                                   \\
     & 6 = 78 - 2 \cdot (192 - 2 \cdot 78)                                   \\
     & 6 = (270 - 1 \cdot 192) - 2 \cdot (192 - 2 \cdot (270 - 1 \cdot 192)) \\
     & 6 = 270 -  1 \cdot 192 - 2 \cdot 192 + 4 \cdot 270 - 4 \cdot 192      \\
     & 6 = 5 \cdot 270 - 7 \cdot 192
\end{align*}
\par Portanto, $s = 5, t = -7$.
\vspace{6pt}\\

\section{DEFINIÇÃO 5 - Inverso Multiplicativo Modular}
Este é um conceito essencial que se relaciona com o conceito de congruência linear (\textbf{definição 6}).\\\\
Sejam $a,m \in \mathbb{Z}$. O inverso multiplicativo modular de $a \bmod m$ é o inteiro $x$ tal que:
\[
    ax \equiv 1 \pmod m
\]

\subsection{Condição de existência}
O inverso multiplicativo modular de $a \bmod m$ existe se, e somente se $mdc(a,m) = 1$, isto é, se $a$ e $m$ forem \textbf{coprimos} ou \textbf{primos entre si}.\\

\subsection{Métodos para encontrar}
É possível encontrar o inverso multiplicativo de $a \bmod m$ facilmente usando o \textbf{teorema 2 - teorema de Bézout}.
\\\\Ao escrever o $mdc(a,m)$ como uma combinação linear de $a$ e $m$, o coeficiente de $a$ é o seu inverso multiplicativo.

\subsection*{Exemplos}
\textbf{1.} Encontrar o inverso multiplicativo de $3 \bmod 7$.

\begin{align*}
     & 3x \equiv 1 \pmod 7                                             \\
    \\
     & \text{Primeiro, precisamos calcular $mdc(3,7)$.}                \\
    \\
     & 7 = 3 \cdot 2 + 1                                               \\
     & 3 = 1 \cdot 3 + 0                                               \\
    \\
     & \text{Como $mdc(3,7) = 1$, o inverso multiplicativo existe.}    \\
     & \text{Agora, escrevemos 1 como uma combinação linear de 3 e 7.} \\
    \\
     & 1 = 1 \cdot 7 - 2 \cdot 3                                       \\
    \\
     & \text{O coeficiente de 3 é -2, então $x = -2$.}
\end{align*}

Como o inverso multiplicativo encontrado é um número negativo, podemos fazer a operação $x \bmod m$ para encontrar um inverso multiplicativo positivo (o que é uma boa prática).

\[
    -2 \bmod 7 = 5
\]

Logo, o inverso multiplicativo que procuramos é $\mathbf{5}$.\\

\textbf{Obs:} Para encontrar um inverso multiplicativo positivo também é possível somar $m$ a $x$ até que $x$ seja maior ou igual a 1.

\section{DEFINIÇÃO 6 - Congruência Linear}
Uma congruência linear é uma equação na forma $ax \equiv b \pmod m$. Uma congruência linear tem solução se, e somente se $mdc(a,m) \mid b$.\\
\\
Para resolver a congruência linear, é necessário seguir os seguintes passos:

\begin{enumerate}

    \item Encontrar o inverso multiplicativo de $a$ -- denotado por $\overline{a}$ ou $a^{-1}$ -- utilizando o método descrito na \textbf{definição 5}.
          \[
              a \overline{a} \equiv 1 \pmod m
          \]

    \item Multiplicar os dois lados da congruência por $\overline{a}$.
          \[
              \overline{a}ax \equiv \overline{a}b \pmod m
          \]
    \item Simplificando, o resultado fica:
          \[
              x \equiv \overline{a}b \pmod m
          \]
    \item Se $\overline{a}b < 1$ ou $\overline{a}b \geq m$, é necessário executar a operação $(\overline{a}b \bmod m)$ para encontrar a menor congruência natural.

\end{enumerate}

\subsection*{Exemplos}
\textbf{1.} Calcular $17x \equiv 82 \pmod{11}$

\begin{align*}
     & 17\overline{a} \equiv 1 \pmod{11}                                \\
    \\
     & 17 = 11 \cdot 1 + 6                                              \\
     & 11 = 6 \cdot 1 + 5                                               \\
     & 6 = 5 \cdot 1 + 1                                                \\
    \\
     & 1 = 6 - 1 \cdot 5                                                \\
     & 1 = 6 - 1 \cdot (11 - 1 \cdot 6)                                 \\
     & 1 = (17 - 1 \cdot 11) - 1 \cdot (11 - 1 \cdot (17 - 1 \cdot 11)) \\
     & 1 = 17 - 1 \cdot 11 - 1 \cdot 11 + 1 \cdot 17 -1 \cdot 11        \\
     & 1 = 2 \cdot 17 - 3 \cdot 11                                      \\
    \\
     & \overline{a} = 2                                                 \\
    \\
     & 2 \cdot 17x \equiv 2 \cdot 82 \pmod {11}                         \\
     & 34x \equiv 164 \pmod {11}                                        \\
     & \boxed{x \equiv 10 \pmod {11}}                                   \\
\end{align*}

Também é possível escrever a solução na forma de um conjunto solução, usando as definições 2 e 1 \textbf{(módulo e divisibilidade)}, respectivamente:
\begin{align*}
    11  & \mid x - 10                                                \\
    11k & = x - 10                                                   \\
    x   & = 11k + 10                                                 \\
    \\
    S   & = \{x \in \mathbb{Z} \mid x = 10 + 11k, k \in \mathbb{Z}\}
\end{align*}

\section*{TEOREMA 3 - Teorema Chinês do Resto}
O teorema chinês do resto é um teorema que pode ser usado para resolver sistemas de congruências lineares do tipo:
\[
    \begin{cases}
        x \equiv a_1 \pmod{m_1}         \\
        x \equiv a_2 \pmod{m_2}         \\
        \dots                           \\
        x \equiv a_{n-1} \pmod{m_{n-1}} \\
        x \equiv a_n \pmod{m_n}
    \end{cases}
\]
\textbf{Obs:} A solução só existe se $mdc(m_i, m_j) = 1$ para todo $i \neq j$.

\subsection*{Algoritmo de solução}
\begin{enumerate}
    \item Calcular o módulo total ($M$):
          \[
              M = m_1 \cdot m_2 \cdot \dots \cdot m_n
          \]
    \item Calcular os $M_i$:
          \[
              \text{$M_i$ é o produto de todos os módulos do sistema, excluindo o módulo $m_i$.}
          \]
    \item Encontrar o inverso $y_i$:
          \[
              M_i \cdot y_i \equiv 1 \pmod{m_i}
          \]
    \item Calcular a solução ($x$):
          \[
              x \equiv a_1 M_1 y_1 + a_2 M_2 y_2 + \dots + a_n M_n y_n \pmod M
          \]

          \textbf{Obs:} Na maioria das vezes, a solução final será o resto da divisão dessa soma por $M$.
\end{enumerate}

\subsection*{Exemplos}
\textbf{1.} Calcular a solução de:
\[
    \begin{cases}
        x \equiv 2 \pmod 3 \\
        x \equiv 3 \pmod 5 \\
        x \equiv 2 \pmod 7 \\
    \end{cases}
\]

\begin{enumerate}
    \item Módulo total ($\mathbf{M}$):
          \begin{align*}
              M & = 3 \cdot 5 \cdot 7 \\
              M & = 105
          \end{align*}
    \item Cálculo dos $\mathbf{M_i}$:
          \begin{align*}
              M_1 & = 5 \cdot 7 = 35 \\
              M_2 & = 3 \cdot 7 = 21 \\
              M_3 & = 3 \cdot 5 = 15 \\
          \end{align*}
    \item Inverso $\mathbf{y_i}$:
          \begin{align*}
              35 \cdot y_1 & \equiv 1 \pmod 3 \\
              21 \cdot y_2 & \equiv 1 \pmod 5 \\
              15 \cdot y_3 & \equiv 1 \pmod 7 \\
              \\
              y_1          & = 2              \\
              y_2          & = 1              \\
              y_3          & = 1
          \end{align*}
    \item Solução $\mathbf{x}$:
          \begin{align*}
              x & \equiv (2 \cdot 35 \cdot 2) + (3 \cdot 21 \cdot 1) + (2 \cdot 15 \cdot 1) \pmod{105} \\
              x & \equiv 140 + 63 + 30 \pmod {105}                                                     \\
              x & \equiv 233 \pmod {105}                                                               \\
          \end{align*}
          \[
              \boxed{
                  x \equiv 23 \pmod {105}
              }
          \]
\end{enumerate}

\section*{TEOREMA 4 - Pequeno Teorema de Fermat}
Este é um teorema desenvolvido pelo matemático francês Pierre de Fermat, e possibilita a simplificação de cálculos com potências grandes na aritmética modular.\\
O teorema pode ser enunciado de duas formas:

\subsubsection*{Primeira forma}
Seja $p$ um número primo e $a \in \mathbb{Z}$. Então:
\[
    a^p \equiv a \pmod{p}
\]
Ou seja, se elevarmos $a$ à potência do primo $p$ e depois dividirmos o resultado por $p$, o resto é igual ao resto da divisão de $a$ por $p$.

\subsubsection*{Segunda forma}
Esta é a forma mais usada.\\
Seja $p$ um número primo e $a \in \mathbb{Z}$ tal que $p \nmid a$ (ou seja, $mdc(a,p) = 1$). Então:
\[
    a^{p-1} \equiv 1 \pmod {p}
\]
Ou seja, se elevarmos $a$ à potência de $p-1$ e depois dividirmos o resultado por $p$, o resultado vai sempre ser igual a $1$.

\subsection*{Exemplos}
\textbf{1.} Calcular $2^{23} \bmod 5$ (ou seja, o resto da divisão de $2^{23}$ por $5$).

\[
    \text{Como 5 é um número primo, é possível utilizar o pequeno teorema de Fermat.}
\]

\setcounter{equation}{0}
\begin{align}
    2^{5-1}                 & \equiv 1 \pmod{5}                 \\
    2^{4}                   & \equiv 1 \pmod {5}                \\
    (2^{4})^{5}             & \equiv 1^{5} \pmod {5}            \\
    (2^{4})^{5} \cdot 2^{3} & \equiv 1^{5} \cdot 2^{3} \pmod{5} \\
    2^{23}                  & \equiv 8 \pmod{5}                 \\
    2^{23}                  & \equiv 3 \pmod {5}
\end{align}
\[
    \text{Portanto, o resto da divisão de $2^{23}$ por 5 é igual a 3.}
\]

\vspace{14pt}
\hrulefill
\vspace{20pt}

\noindent
\colorbox{yellow!30}{
    \begin{minipage}{\linewidth-2\fboxsep-2\fboxrule}
        \textbf{NOTA:}
        Aqui acabam (supostamente) os conteúdos da primeira prova da unidade 2 (que foi dividida em duas partes, sendo a primeira no dia \textbf{15 de outubro}).
    \end{minipage}
}

\end{document}