\section*{TEOREMA 4 - Pequeno Teorema de Fermat}
Este é um teorema desenvolvido pelo matemático francês Pierre de Fermat, e possibilita a simplificação de cálculos com potências grandes na aritmética modular.\\
O teorema pode ser enunciado de duas formas:

\subsubsection*{Primeira forma}
Seja $p$ um número primo e $a \in \mathbb{Z}$. Então:
\[
    a^p \equiv a \pmod{p}
\]
Ou seja, se elevarmos $a$ à potência do primo $p$ e depois dividirmos o resultado por $p$, o resto é igual ao resto da divisão de $a$ por $p$.

\subsubsection*{Segunda forma}
Esta é a forma mais usada.\\
Seja $p$ um número primo e $a \in \mathbb{Z}$ tal que $p \nmid a$ (ou seja, $mdc(a,p) = 1$). Então:
\[
    a^{p-1} \equiv 1 \pmod {p}
\]
Ou seja, se elevarmos $a$ à potência de $p-1$ e depois dividirmos o resultado por $p$, o resultado vai sempre ser igual a $1$.

\subsection*{Exemplos}
\textbf{1.} Calcular $2^{23} \bmod 5$ (ou seja, o resto da divisão de $2^{23}$ por $5$).

\[
    \text{Como 5 é um número primo, é possível utilizar o pequeno teorema de Fermat.}
\]

\setcounter{equation}{0}
\begin{align}
    2^{5-1}                 & \equiv 1 \pmod{5}                 \\
    2^{4}                   & \equiv 1 \pmod {5}                \\
    (2^{4})^{5}             & \equiv 1^{5} \pmod {5}            \\
    (2^{4})^{5} \cdot 2^{3} & \equiv 1^{5} \cdot 2^{3} \pmod{5} \\
    2^{23}                  & \equiv 8 \pmod{5}                 \\
    2^{23}                  & \equiv 3 \pmod {5}
\end{align}
\[
    \text{Portanto, o resto da divisão de $2^{23}$ por 5 é igual a 3.}
\]