\section*{TEOREMA 3 - Teorema Chinês do Resto}
O teorema chinês do resto é um teorema que pode ser usado para resolver sistemas de congruências lineares do tipo:
\[
    \begin{cases}
        x \equiv a_1 \pmod{m_1}         \\
        x \equiv a_2 \pmod{m_2}         \\
        \dots                           \\
        x \equiv a_{n-1} \pmod{m_{n-1}} \\
        x \equiv a_n \pmod{m_n}
    \end{cases}
\]
\textbf{Obs:} A solução só existe se $mdc(m_i, m_j) = 1$ para todo $i \neq j$.

\subsection*{Algoritmo de solução}
\begin{enumerate}
    \item Calcular o módulo total ($M$):
          \[
              M = m_1 \cdot m_2 \cdot \dots \cdot m_n
          \]
    \item Calcular os $M_i$:
          \[
              \text{$M_i$ é o produto de todos os módulos do sistema, excluindo o módulo $m_i$.}
          \]
    \item Encontrar o inverso $y_i$:
          \[
              M_i \cdot y_i \equiv 1 \pmod{m_i}
          \]
    \item Calcular a solução ($x$):
          \[
              x \equiv a_1 M_1 y_1 + a_2 M_2 y_2 + \dots + a_n M_n y_n \pmod M
          \]

          \textbf{Obs:} Na maioria das vezes, a solução final será o resto da divisão dessa soma por $M$.
\end{enumerate}

\subsection*{Exemplos}
\textbf{1.} Calcular a solução de:
\[
    \begin{cases}
        x \equiv 2 \pmod 3 \\
        x \equiv 3 \pmod 5 \\
        x \equiv 2 \pmod 7 \\
    \end{cases}
\]

\begin{enumerate}
    \item Módulo total ($\mathbf{M}$):
          \begin{align*}
              M & = 3 \cdot 5 \cdot 7 \\
              M & = 105
          \end{align*}
    \item Cálculo dos $\mathbf{M_i}$:
          \begin{align*}
              M_1 & = 5 \cdot 7 = 35 \\
              M_2 & = 3 \cdot 7 = 21 \\
              M_3 & = 3 \cdot 5 = 15 \\
          \end{align*}
    \item Inverso $\mathbf{y_i}$:
          \begin{align*}
              35 \cdot y_1 & \equiv 1 \pmod 3 \\
              21 \cdot y_2 & \equiv 1 \pmod 5 \\
              15 \cdot y_3 & \equiv 1 \pmod 7 \\
              \\
              y_1          & = 2              \\
              y_2          & = 1              \\
              y_3          & = 1
          \end{align*}
    \item Solução $\mathbf{x}$:
          \begin{align*}
              x & \equiv (2 \cdot 35 \cdot 2) + (3 \cdot 21 \cdot 1) + (2 \cdot 15 \cdot 1) \pmod{105} \\
              x & \equiv 140 + 63 + 30 \pmod {105}                                                     \\
              x & \equiv 233 \pmod {105}                                                               \\
          \end{align*}
          \[
              \boxed{
                  x \equiv 23 \pmod {105}
              }
          \]
\end{enumerate}