\section*{TEOREMA 1}
Sejam $a,b,m \in \mathbb{Z}$, com $m > 0$. $a$ é congruente a $b$ módulo $m$ se, e somente se $a \bmod m = b \bmod m$.
\[
     a \equiv b \pmod m \iff a \bmod m = b \bmod m \text{.}
\]
\subsection*{Prova ($\implies$)}
Suponha $a, b, m \in \mathbb{Z}$ tal que $a \equiv b \pmod m$.\\
Pelas definições 2 e 1, respectivamente, temos que:
\begin{flalign*}
      & m \mid a -b                          \\
      & mk = a-b                             \\
      & a = b + mk \quad \text{\textbf{(I)}}
\end{flalign*}
Pela definição do resto, ao dividir $b$ por $m$, temos:
\[
     b = q_b \cdot m + r_b \quad \text{\textbf{(II)}}
\]
Onde $q_b \in \mathbb{Z}$ e $r_b = b \bmod m$, com $0 \leq r_b < m$.\\
Substituindo \textbf{(II)} em \textbf{(I)}:
\begin{flalign*}
      & a = (q_b \cdot m + r_b) + mk \\
      & a = mq_b + mk + r_b          \\
      & a = m\cdot(q_b + k) + r_b
\end{flalign*}
Tome $q_a = q_b + k$. Como $q_a \in \mathbb{Z}$ e $0 \leq r_b < m$, podemos dizer que $r_b$ é o resto da divisão de $a$ por $m$, isto é, $a \bmod m = r_b$.\\
Como $r_b = b \bmod m$, temos que:
\[
     a \bmod m = b \bmod m \text{.}
\]
\subsection*{Prova ($\impliedby$)}
Suponha $a, b, m \in \mathbb{Z}$ tal que $a \bmod m = b \bmod m$.
Seja $r$ o valor comum do resto, de forma que:

\[
     r = a\bmod m = b \bmod m
\]

Pela definição do resto, podemos escrever $a$ e $b$ como:
\begin{flalign*}
      & a = q_a \cdot m + r \\
      & b = q_b \cdot m + r
\end{flalign*}

Onde $q_a, q_b \in \mathbb{Z}$ e $0 \leq r < m$.
Dessa forma, a diferença $a - b$ fica desta forma:
\begin{flalign*}
      & a - b = (q_a \cdot m + r) - (q_b \cdot m + r) \\
      & a - b = q_a \cdot m + r - q_b \cdot m - r     \\
      & a - b = m \cdot (q_a-q_b)
\end{flalign*}

Seja $k = q_a-q_b$. Como $q_a \in \mathbb{Z}$ e $q_b \in \mathbb{Z}$ temos que $k \in \mathbb{Z}$. Portanto:

\[
     a-b = mk
\]

Pela definição 1 e 2, respectivamente, temos que:
\begin{flalign*}
      & m \mid a-b         \\
      & a \equiv b \pmod m
\end{flalign*}

\begin{flalign*}
      &  & \blacksquare
\end{flalign*}